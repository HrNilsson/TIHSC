%!TEX root = TIHSC_Project_main.tex
\chapter{Requirement Specification}
The system requirements is divided into two sections; functional ad non-functional requirements. A standard approach to capture the functional requirements would be to use SysML/UML Use Cases. However due to the simplicity of this project it has been chosen to keep the functional requirements simple and only use one line statements. 

\section{Functional requirements}
\begin{enumerate}[label=1.\arabic*]
	\item The system shall be able to read in an image.
	\item The system shall be able to apply Texton filtering to an image.
	\item The system shall be able to output the filtered image.
\end{enumerate}


\section{Non-Functional requirements}
\begin{enumerate}[label=2.\arabic*]
	\item The Texton filtering shall be done within a minute.
	\label{req:timeReq}
	\item The resolution of the image shall be 512 x 512 pixels.
	\item The input image shall be 8 bit (gray scale).
	\item The output image shall be 24 bit (RGB).
\end{enumerate}